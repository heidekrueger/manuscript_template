%*************************************************************
%*****    Stefan's LaTeX Template
%****
%*****    I took the original version of this template from 
%****     https://arxiv.org/abs/1608.07310, i.e. from Panayotis Mertikopoulos.
%****     I don't know, where it originated.
%****
%****     This version is customized by and for Stefan Heidekrüger.
%****
%*************************************************************

%!TEX TS-program =  pdflatex


%*************************************************************
%*****    DOCUMENT CLASS
%*************************************************************
\documentclass[reqno]{amsart}



%*************************************************************
%*****    PACKAGES
%*************************************************************

%----------------------------------------------------------------------
%% Basic math input
%----------------------------------------------------------------------
\usepackage{amsmath}
\usepackage{amssymb}
\usepackage{amsfonts}
\usepackage{amsthm}
\usepackage[foot]{amsaddr}
\usepackage{mathtools}
\mathtoolsset{%
%showonlyrefs,	% to show only referenced equations
}


%----------------------------------------------------------------------
%% Fonts and alphabets (beware of conflicts)
%----------------------------------------------------------------------
\usepackage[utf8]{inputenc}
\usepackage[T1]{fontenc}


%% Minion Pro
%----------------------------------------------------------------------
%\linespread{1.05}
%\usepackage[lf,minionint,openg,footnotefigures]{MinionPro}
%\usepackage[bib,eqno,enum]{tabfigures}
%\input glyphtounicode
%\pdfgentounicode=1


%% Baskerville
%----------------------------------------------------------------------
%\pdfmapfile{+baskervillef.map}
%\usepackage[sups,p,theoremfont]{baskervillef}
%\usepackage[varqu,varl,var0]{inconsolata}
%\usepackage[scale=.95,type1]{cabin}
%\usepackage[baskerville,vvarbb]{newtxmath}


%% Libertine
%----------------------------------------------------------------------
\usepackage[sf,mono=false]{libertine}
%\usepackage[libertine,libaltvw,cmintegrals]{newtxmath}


%% Garamond
%----------------------------------------------------------------------
%\usepackage[full]{textcomp}
%\usepackage{garamondx}
%\usepackage[garamondx,cmintegrals,cmbraces]{newtxmath}
%\useosf

%\usepackage[urw-garamond]{mathdesign}
%\usepackage{garamondx}

%\usepackage[cmintegrals,cmbraces]{newtxmath}
%\usepackage{ebgaramond-maths}
%\makeatletter
%\DeclareSymbolFont{cmletters}{OML}{cmr}{m}{it}
%	\SetSymbolFont{cmletters}{bold}{OML}{cmr}{b}{it}
%	\re@DeclareMathSymbol{\leftharpoonup}{\mathrel}{cmletters}{"28}
%	\re@DeclareMathSymbol{\leftharpoondown}{\mathrel}{cmletters}{"29}
%	\re@DeclareMathSymbol{\rightharpoonup}{\mathrel}{cmletters}{"2A}
%	\re@DeclareMathSymbol{\rightharpoondown}{\mathrel}{cmletters}{"2B}
%	\re@DeclareMathSymbol{\triangleleft}{\mathbin}{cmletters}{"2F}
%	\re@DeclareMathSymbol{\triangleright}{\mathbin}{cmletters}{"2E}
%	\re@DeclareMathSymbol{\partial}{\mathord}{cmletters}{"40}
%	\re@DeclareMathSymbol{\flat}{\mathord}{cmletters}{"5B}
%	\re@DeclareMathSymbol{\natural}{\mathord}{cmletters}{"5C}
%	\re@DeclareMathSymbol{\star}{\mathbin}{cmletters}{"3F}
%	\re@DeclareMathSymbol{\smile}{\mathrel}{cmletters}{"5E}
%	\re@DeclareMathSymbol{\frown}{\mathrel}{cmletters}{"5F}
%	\re@DeclareMathSymbol{\sharp}{\mathord}{cmletters}{"5D}
%	\re@DeclareMathAccent{\vec}{\mathord}{cmletters}{"7E}
%\makeatother


%% Charter
%----------------------------------------------------------------------
%\usepackage[bitstream-charter]{mathdesign}

%% Palatino (uncomment one, not both)
%----------------------------------------------------------------------
%\linespread{1.05}
%\usepackage{newpxtext,newpxmath}
%\usepackage[sc]{mathpazo}

%% Times
%----------------------------------------------------------------------
%\usepackage[varg]{txfonts}
%\let\mathbb=\varmathbb


%% Blackboard bold
%----------------------------------------------------------------------
\usepackage{dsfont}
%\let\mathbb=\mathds


%% Sans serif font
%----------------------------------------------------------------------
%\usepackage[lf,scaled=.92]{carlito}
%\usepackage[lf,scaled=.92]{sourcesanspro}


%% Typewriter font
%----------------------------------------------------------------------
%\usepackage{sourcecodepro}


%% Math alphabets
%----------------------------------------------------------------------
% For choosing style of math alphabets.
% see http://ctan.math.washington.edu/tex-archive/macros/latex/contrib/mathalpha/doc/mathalpha-doc.pdf

\usepackage[%
cal=cm,
%bb=fourier,
%scr=euler,
%frak=euler
]
{mathalfa}


%----------------------------------------------------------------------
%% Colors
%----------------------------------------------------------------------
\usepackage[dvipsnames,svgnames]{xcolor}
\colorlet{MyBlue}{DodgerBlue!60!Black}
\colorlet{MyGreen}{DarkGreen!85!Black}


%----------------------------------------------------------------------
%% Figures, Tables and Graphics
%----------------------------------------------------------------------
\usepackage[font=small,labelfont=bf]{caption}
\captionsetup[algorithm]{labelfont={bf,sf,normalsize},labelsep=period}
\usepackage{subfigure}
\usepackage{tikz}
\usetikzlibrary{arrows,calc,patterns}
\usepackage{svg}

\usepackage{booktabs} % For formal tables


%----------------------------------------------------------------------
%% Miscellaneous
%----------------------------------------------------------------------

% add `nolist` option if printing not desired.
\usepackage[printonlyused,withpage]{acronym} % define acronyms an reference them with
\usepackage{latexsym}
\usepackage{nicefrac}
\usepackage{paralist}
\usepackage{wasysym}
\usepackage{xspace}
% used to check dynamically whether to print acronym section
\makeatletter
\@ifpackagewith{acronym}{nolist}
    {\newcommand{\makeAcronymSection}{}}
    {\newcommand{\makeAcronymSection}{\section{Acronyms}}} %print
\makeatother
%----------------------------------------------------------------------
%% References
%----------------------------------------------------------------------
\usepackage[authoryear,compress]{natbib}
\def\bibfont{\footnotesize}
\def\bibsep{\smallskipamount}
\def\bibhang{24pt}
%\def\BIBand{and}
%\def\newblock{\ }
%\bibpunct[, ]{[}{]}{,}{n}{}{,}

\newcommand{\citef}{\cite}
\newcommand{\citefp}{\citep}
\newcommand{\citepos}[1]{\citeauthor{#1}'s~\textpar{\citeyear{#1}}}
%\newcommand{\citef}[2][]{\citeauthor{#2} \cite[#1]{#2}}
%\newcommand{\citefp}[2][]{\textup(\citef[#1]{#2}\textup)}


%----------------------------------------------------------------------
%% Hyperlinks
%----------------------------------------------------------------------
\usepackage{hyperref}
\hypersetup{
colorlinks=true,
linktocpage=true,
%pdfstartpage=1,
pdfstartview=FitH,
breaklinks=true,
pdfpagemode=UseNone,
pageanchor=true,
pdfpagemode=UseOutlines,
plainpages=false,
bookmarksnumbered,
bookmarksopen=false,
bookmarksopenlevel=1,
hypertexnames=true,
pdfhighlight=/O,
%hyperfootnotes=true,
%nesting=true,
%frenchlinks,
urlcolor=MyBlue!60!black,linkcolor=MyBlue!70!black,citecolor=DarkGreen!70!black, % <--- for screen
%urlcolor=black, linkcolor=black, citecolor=black, %pagecolor=black, % <--- for printing
%pagecolor=RoyalBlue,
pdftitle={},
pdfauthor={},
pdfsubject={},
pdfkeywords={},
pdfcreator={pdfLaTeX},
pdfproducer={LaTeX with hyperref}
}


%----------------------------------------------------------------------
%% Cleverefs
%----------------------------------------------------------------------
\numberwithin{equation}{section}  %numberwithin goes before cleverefs when using hyperref
\usepackage[sort&compress,capitalize,nameinlink]{cleveref}
%\crefname{example}{Ex.}{Exs.}
\newcommand{\crefrangeconjunction}{\textendash}
\crefrangeformat{equation}{\upshape(#3#1#4)\textendash(#5#2#6)}


%*************************************************************
%*****    MACROS
%*************************************************************

%----------------------------------------------------------------------
%% Aliases
%----------------------------------------------------------------------
\newcommand{\dd}{\:d}
\newcommand{\del}{\partial}
\newcommand{\eps}{\varepsilon}
\newcommand{\from}{\colon}
\newcommand{\injects}{\hookrightarrow}
\newcommand{\pd}{\partial}
\newcommand{\wilde}{\widetilde}

\newcommand{\mg}{\succ}
\newcommand{\mgeq}{\succcurlyeq}
\newcommand{\ml}{\prec}
\newcommand{\mleq}{\preccurlyeq}


%----------------------------------------------------------------------
%% Boldface
%----------------------------------------------------------------------
\newcommand{\bA}{\mathbf{A}}
\newcommand{\bB}{\mathbf{B}}
\newcommand{\bC}{\mathbf{C}}
\newcommand{\bD}{\mathbf{D}}
\newcommand{\bE}{\mathbf{E}}
\newcommand{\bF}{\mathbf{F}}
\newcommand{\bG}{\mathbf{G}}
\newcommand{\bH}{\mathbf{H}}
\newcommand{\bI}{\mathbf{I}}
\newcommand{\bJ}{\mathbf{J}}
\newcommand{\bK}{\mathbf{K}}
\newcommand{\bL}{\mathbf{L}}
\newcommand{\bM}{\mathbf{M}}
\newcommand{\bN}{\mathbf{N}}
\newcommand{\bO}{\mathbf{O}}
\newcommand{\bP}{\mathbf{P}}
\newcommand{\bQ}{\mathbf{Q}}
\newcommand{\bR}{\mathbf{R}}
\newcommand{\bS}{\mathbf{S}}
\newcommand{\bT}{\mathbf{T}}
\newcommand{\bU}{\mathbf{U}}
\newcommand{\bV}{\mathbf{V}}
\newcommand{\bW}{\mathbf{W}}
\newcommand{\bX}{\mathbf{X}}
\newcommand{\bY}{\mathbf{Y}}
\newcommand{\bZ}{\mathbf{Z}}

\newcommand{\ba}{\mathbf{a}}
\newcommand{\boldb}{\mathbf{b}}
\newcommand{\bc}{\mathbf{c}}
\newcommand{\boldd}{\mathbf{d}}
\newcommand{\be}{\mathbf{e}}
\newcommand{\boldf}{\mathbf{f}}
\newcommand{\bg}{\mathbf{g}}
\newcommand{\bh}{\mathbf{h}}
\newcommand{\bi}{\mathbf{i}}
\newcommand{\bj}{\mathbf{j}}
\newcommand{\bk}{\mathbf{k}}
\newcommand{\bl}{\mathbf{l}}
\newcommand{\bm}{\mathbf{m}}
\newcommand{\bn}{\mathbf{n}}
\newcommand{\bo}{\mathbf{o}}
\newcommand{\bp}{\mathbf{p}}
\newcommand{\bq}{\mathbf{q}}
\newcommand{\br}{\mathbf{r}}
\newcommand{\bs}{\mathbf{s}}
\newcommand{\bt}{\mathbf{t}}
\newcommand{\bu}{\mathbf{u}}
\newcommand{\bv}{\mathbf{v}}
\newcommand{\bw}{\mathbf{w}}
\newcommand{\bx}{\mathbf{x}}
\newcommand{\by}{\mathbf{y}}
\newcommand{\bz}{\mathbf{z}}


%----------------------------------------------------------------------
%% Fields
%----------------------------------------------------------------------
\newcommand{\F}{\mathbb{F}}
\newcommand{\C}{\mathbb{C}}
\newcommand{\R}{\mathbb{R}}
\newcommand{\Q}{\mathbb{Q}}
\newcommand{\Z}{\mathbb{Z}}
\newcommand{\N}{\mathbb{N}}


%----------------------------------------------------------------------
%% Operators
%----------------------------------------------------------------------



% Original List
% a * means limits are typeset below, not on the right
%      at least in display mode
\DeclareMathOperator*{\aff}{aff}
\DeclareMathOperator*{\argmax}{arg\,max}
\DeclareMathOperator*{\argmin}{arg\,min}
\DeclareMathOperator*{\intersect}{\bigcap}
\DeclareMathOperator*{\lip}{lip}
\DeclareMathOperator*{\vspan}{span}
\DeclareMathOperator*{\union}{\bigcup}

\DeclareMathOperator{\bd}{bd}
\DeclareMathOperator{\bigoh}{\mathcal{O}}
\DeclareMathOperator{\card}{card}
\DeclareMathOperator{\cl}{cl}
\DeclareMathOperator{\diag}{diag}
\DeclareMathOperator{\diam}{diam}
\DeclareMathOperator{\dist}{dist}
\DeclareMathOperator{\dom}{dom}
\DeclareMathOperator{\eig}{eig}
\DeclareMathOperator{\ess}{ess}
\DeclareMathOperator{\ex}{\mathbb{E}}
\DeclareMathOperator{\grad}{\nabla}
\DeclareMathOperator{\hess}{Hess}
\DeclareMathOperator{\hull}{\Delta}
\DeclareMathOperator{\im}{im}
\DeclareMathOperator{\ind}{ind}
\DeclareMathOperator{\intr}{int}
\DeclareMathOperator{\one}{\mathds{1}}
\DeclareMathOperator{\proj}{pr}
\DeclareMathOperator{\prob}{\mathbb{P}}
\DeclareMathOperator{\rank}{rank}
\DeclareMathOperator{\rel}{rel}
\DeclareMathOperator{\relint}{ri}
\DeclareMathOperator{\sign}{sgn}
\DeclareMathOperator{\supp}{supp}
\DeclareMathOperator{\tr}{tr}
\DeclareMathOperator{\var}{var}
\DeclareMathOperator{\vol}{vol}


%----------------------------------------------------------------------
%% Delimiters
%----------------------------------------------------------------------
\providecommand\given{} % provides an empty command for the delimiters below

\DeclarePairedDelimiter{\braces}{\{}{\}}
\DeclarePairedDelimiter{\bracks}{[}{]}
\DeclarePairedDelimiter{\parens}{(}{)}

\DeclarePairedDelimiter{\abs}{\lvert}{\rvert}
\DeclarePairedDelimiter{\norm}{\lVert}{\rVert}
\DeclarePairedDelimiter{\dnorm}{\lVert}{\rVert_{\ast}}
%\newcommand{\dnorm}[1]{\norm{#1}_{\ast}}

\DeclarePairedDelimiter{\ceil}{\lceil}{\rceil}
\DeclarePairedDelimiter{\floor}{\lfloor}{\rfloor}
\DeclarePairedDelimiter{\clip}{[}{]}
\DeclarePairedDelimiter{\negpart}{[}{]_{-}}
\DeclarePairedDelimiter{\pospart}{[}{]_{+}}

\DeclarePairedDelimiter{\bra}{\langle}{\rvert}
\DeclarePairedDelimiter{\ket}{\lvert}{\rangle}
%\DeclarePairedDelimiterX{\braket}[2]{\langle}{\rangle}{#1\mathopen{}\hspace{1pt}\delimsize\vert\hspace{1pt}\mathopen{}#2}
\DeclarePairedDelimiterX{\braket}[2]{\langle}{\rangle}{#1,#2}

\DeclarePairedDelimiterX{\inner}[2]{\langle}{\rangle}{#1,#2}
\DeclarePairedDelimiterX{\setdef}[2]{\{}{\}}{#1:#2}

% command \Set with scaled | using \given
\newcommand\SetSymbol[1][]{%
\nonscript\:#1\vert
\allowbreak
\nonscript\:
\mathopen{}}
\DeclarePairedDelimiterX\Set[1]\{\}{%
\renewcommand\given{\SetSymbol[\delimsize]}
#1
}

% Probability and conditional probability

\DeclarePairedDelimiterXPP{\probof}[1]{\prob}{(}{)}{}{%
\renewcommand\given{\nonscript\,\delimsize\vert\nonscript\,\mathopen{}}
#1}

% expectation and conditional expectation
\DeclarePairedDelimiterXPP{\exof}[1]{\ex}{[}{]}{}{%
\renewcommand\given{\nonscript\,\delimsize\vert\nonscript\,\mathopen{}}
#1}

% expectation and conditional expectation with explicit subscript
\DeclarePairedDelimiterXPP{\exofsub}[2]{\ex_{#1}}{[}{]}{}{%
    \renewcommand\given{\nonscript\,\delimsize\vert\nonscript\,\mathopen{}}#2}

%    \newcommand{\expect}[2][]{\E_{#1}\left[#2\right]}

%----------------------------------------------------------------------
%% Formatting
%----------------------------------------------------------------------
\newcommand{\dis}{\displaystyle}
\newcommand{\txs}{\textstyle}
\newcommand{\textpar}[1]{\textup(#1\textup)}

\newcommand{\as}{\textpar{\textrm{a.s.}}\xspace}
\renewcommand{\ae}{\textpar{\textrm{a.e.}}\xspace}



%*************************************************************
%*****    EDITING
%*************************************************************
\usepackage[textwidth=30mm]{todonotes}
%\setlength{\marginparwidth}{2cm}

\newcommand{\todoi}[2][] {\vspace{0.5em}\todo[inline, #1]{#2}}

\newcommand{\needref}{{\color{red} \upshape [\textbf{??}]}\xspace}
\newcommand{\attn}{{\color{red} \upshape [\textbf{!!}]}\xspace}
\newcommand{\revise}[1]{{\color{blue}#1}}
%\newcommand{\debug}[1]{{\color{purple}#1}}
\newcommand{\debug}[1]{#1}
\newcommand{\lookout}[1]{{\color{red}#1}}


\newcommand{\start}{\debug 1}
\newcommand{\run}{\debug n}
\newcommand{\iRun}{\debug k}




%*************************************************************
%*****    ENVIRONMENTS
%*************************************************************

%----------------------------------------------------------------------
%% Algorithms
%----------------------------------------------------------------------
%\usepackage{algorithm}
%\usepackage{algorithmic}
\usepackage[boxed, ruled,vlined, linesnumbered, commentsnumbered]{algorithm2e}


%----------------------------------------------------------------------
%% Theorem-like
%----------------------------------------------------------------------
\theoremstyle{plain}
\newtheorem{theorem}{Theorem}
\newtheorem{corollary}[theorem]{Corollary}
\newtheorem*{corollary*}{Corollary}
\newtheorem{lemma}[theorem]{Lemma}
\newtheorem{proposition}[theorem]{Proposition}
\newtheorem{conjecture}[theorem]{\color{orange}Conjecture}


%----------------------------------------------------------------------
%% Definition-like
%----------------------------------------------------------------------
\theoremstyle{definition}
\newtheorem{definition}[theorem]{Definition}
\newtheorem*{definition*}{Definition}
\newtheorem{assumption}[theorem]{Assumption}


%----------------------------------------------------------------------
%% Proofs
%----------------------------------------------------------------------
\newenvironment{Proof}[1][Proof]{\begin{proof}[#1]}{\end{proof}}
\renewcommand\qedsymbol{\small$\blacksquare$}


%----------------------------------------------------------------------
%% Remark-like
%----------------------------------------------------------------------
\theoremstyle{remark}
\newtheorem{remark}{Remark}
\newtheorem*{remark*}{Remark}
\newtheorem*{notation*}{Notational remark}
\newtheorem{example}{Example}


%----------------------------------------------------------------------
%% Numbering
%----------------------------------------------------------------------
\numberwithin{theorem}{section}
\numberwithin{remark}{section}
\numberwithin{example}{section}


%*************************************************************
%*****    FREQUENTLY USED
%*************************************************************

%----------------------------------------------------------------------
%% Linear Algebra
%----------------------------------------------------------------------
\newcommand{\vecspace}{\mathcal{\debug V}}
\newcommand{\dspace}{\vecspace^{\ast}}

\newcommand{\bvec}{e}
\newcommand{\dvec}{\bvec^{\ast}}


%----------------------------------------------------------------------
%% Convex Analysis
%----------------------------------------------------------------------
\newcommand{\body}{\mathcal{D}}
\newcommand{\cvx}{\mathcal{C}}
\newcommand{\feas}{\mathcal{\debug X}}
\newcommand{\intfeas}{\feas^{\circ}}
\newcommand{\primal}{\feas}
\newcommand{\dual}{\mathcal{\debug Y}}

\newcommand{\base}{\debug p}
\newcommand{\notbase}{\debug x}
\newcommand{\baseset}{\mathcal{C}}
\newcommand{\obj}{f}
\newcommand{\sol}{{\debug x}^{\ast}}
\newcommand{\solset}{\feas^{\ast}}
\newcommand{\subd}{\partial}

\newcommand{\cone}{C}
\newcommand{\thull}{T}
\newcommand{\tcone}{\mathrm{TC}}
\newcommand{\tspace}{\mathrm{T}}
\newcommand{\dcone}{\tcone^{\ast}}
\newcommand{\ncone}{\mathrm{NC}}
\newcommand{\pcone}{\mathrm{PC}}


%----------------------------------------------------------------------
%% Bregman
%----------------------------------------------------------------------
\newcommand{\breg}{\debug D}
\newcommand{\mirror}{\debug Q}
\newcommand{\depth}{\debug \Omega}
\newcommand{\fench}{\debug F}

\newcommand{\choice}{\mirror}
\DeclareMathOperator{\Eucl}{\Pi}
\DeclareMathOperator{\logit}{\Lambda}


%----------------------------------------------------------------------
%% Games
%----------------------------------------------------------------------
\newcommand{\game}{\mathcal{\debug G}}
\newcommand{\fingame}{\debug\Gamma}

\newcommand{\play}{\debug i}
\newcommand{\playalt}{\debug j}
\newcommand{\nPlayers}{\debug N}
\newcommand{\players}{\mathcal{\nPlayers}}

\newcommand{\playOne}{\debug A}
\newcommand{\playTwo}{\debug B}

\newcommand{\pure}{\debug\alpha}
\newcommand{\purealt}{\debug\beta}
\newcommand{\nPures}{\debug A}
\newcommand{\pures}{\mathcal{\debug\nPures}}
\newcommand{\peq}{\pure^{\ast}}

\newcommand{\act}{\debug X}
\newcommand{\actions}{\feas}
\newcommand{\acts}{\actions}
\newcommand{\intacts}{\intfeas}

\newcommand{\pay}{\debug u}
\newcommand{\payv}{\debug v}
\newcommand{\loss}{\ell}
\newcommand{\cost}{c}
\newcommand{\pot}{f}

\newcommand{\est}{\hat\payv}
\newcommand{\score}{\debug Y}

\newcommand{\gamefull}{\game(\players,(\acts_{\play})_{\play\in\players},(\pay_{\play})_{\play\in\players})}
\newcommand{\fingamefull}{\fingame(\players,(\pures_{\play})_{\play\in\players},(\pay_{\play})_{\play\in\players})}

\newcommand{\eq}{\sol}
\newcommand{\eqset}{\feas^{\ast}}
\newcommand{\eqnhd}{U^{\ast}}
\newcommand{\payveq}{\payv^{\ast}}
\newcommand{\deq}{y^{\ast}}
\newcommand{\olimit}{x_{\omega}}
\newcommand{\dev}{\debug q}
\newcommand{\val}{\pay^{\ast}}
\newcommand{\gap}{\debug\epsilon}
\newcommand{\length}{\ell}

\DeclareMathOperator{\brep}{BR}
%\DeclareMathOperator{\reg}{Reg}
\newcommand{\reg}{\debug R}


%----------------------------------------------------------------------
%% Sundries
%----------------------------------------------------------------------
\newcommand{\argdot}{\,\cdot\,}
\newcommand{\dkl}{D_{\textup{KL}}}
\newcommand{\filter}{\mathcal{F}}
\newcommand{\gen}{\mathcal{L}}
\newcommand{\interval}{I}
\newcommand{\simplex}{\Delta}
\newcommand{\intsimplex}{\simplex^{\!\circ}}
\newcommand{\set}{\mathcal{S}}
\newcommand{\step}{\debug\gamma}
\newcommand{\temp}{\eta}

\newcommand{\nodes}{\mathcal{V}}
\newcommand{\edge}{e}
\newcommand{\nEdges}{E}
\newcommand{\edges}{\mathcal{\nEdges}}

\newcommand{\source}{s}
\newcommand{\sources}{\mathcal{S}}
\newcommand{\sink}{d}
\newcommand{\sinks}{\mathcal{D}}

\newcommand{\resource}{r}
\newcommand{\nResources}{R}
\newcommand{\resources}{\mathcal{\nResources}}
\newcommand{\load}{w}

\newcommand{\route}{\pure}
\newcommand{\routes}{\pures}

\newcommand{\hessmat}{H^{\game}}
\newcommand{\vbound}{\debug V_{\ast}}
\newcommand{\noise}{\debug\xi}
\newcommand{\noisedev}{\debug\sigma}
\newcommand{\noisevar}{\noisedev^{2}}
\newcommand{\snoise}{\psi}
%\newcommand{\lip}{L}
\newcommand{\semiflow}{\debug\Phi}

\newcommand{\strong}{\debug K}
\renewcommand{\sharp}{\debug c}



%*************************************************************
%*****    AUTHOR-SPECIFIC COMMANDS
%*************************************************************

%----------------------------------------------------------------------
%% PM
%----------------------------------------------------------------------
\newcommand{\PM}[1]{\todo[color=DodgerBlue!20!LightGray,author=\textbf{PM},inline]{\small #1\\}}
\newcommand{\negspace}{\!\!\!}

\newcommand{\ie}{i.e.,\xspace}
\newcommand{\eg}{e.g.,\xspace}




%*************************************************************
%*****    MAIN DOCUMENT
%*************************************************************
\begin{document}



%*************************************************************
%*****    FRONT MATTER AND METADATA
%*************************************************************


%----------------------------------------------------------------------
%%% TITLE & AUTHORS
%----------------------------------------------------------------------
\title
[NPGA Convergence in Auctions]
{When does NGPA converge in Bayesian Auction Games?}


\author
[S.~Heidekrüger]
{Stefan Heidekrüger$^{1}$}
\address{$^{1}$
Department of Informatics, Technical University of Munich}
\email{\href{mailto:stefan.heidekrueger@in.tum.de}{stefan.heidekrueger@in.tum.de}}
% add more authors here with sequential address number

%----------------------------------------------------------------------
%%% THANKS
%----------------------------------------------------------------------
% !TEX root = ./Main.tex
%
%
%\thanks{Some thanks text.}



%----------------------------------------------------------------------
%%% KEYWORDS
%----------------------------------------------------------------------
% don't know what this line does exactly
%\subjclass[2010]{Primary 91A26, 90C15; secondary 90C33, 68Q32.}
\keywords{%
Bayesian Game;
Equilibrium Computation;
No regret algorithms;
Reinforcement Learning.
}

  
%----------------------------------------------------------------------
%%% ACRONYMS
%----------------------------------------------------------------------


% see documentation of acronyms package for mor detail
% in text: \ac (first time full name + abbrev., afterwards only abbrev), \Ac (capitalize), \acf (print full name + acronym), \Acf, \acs (short version), \acl (full name without acronym), \Acl, \acp (as \ac but for plural), \Acp, \acfp, \Acfp, \aclp, \Aclp, \acfi (full name in italic, short form in unshaped), \Acfi
% \iac, \Iac adds appropriate indefinite article (e.g. a Nash Eq, but an NE)
% adding a star to any of these \ac* etc does NOT mark them as used for later in the text.
% plural forms will usually be formed by adding s unless otherwise specified.

\newcommand{\acdef}[1]{\textit{\acl{#1}} \textup{(\acs{#1})}\acused{#1}}
\newcommand{\acdefp}[1]{\emph{\aclp{#1}} \textup(\acsp{#1}\textup)\acused{#1}}
\newcommand{\acli}[1]{\textit{\acl{#1}}}


%% definition of actual acronyms has been moved to
%% Acronyms.tex


%----------------------------------------------------------------------
%%% ABSTRACT
%----------------------------------------------------------------------
\begin{abstract}
%----------------------------------------------------------------------
%%% ABSTRACT
%----------------------------------------------------------------------
% !TEX root = ./Main.tex
%
%


This manuscript is menat to outline and develop theory about convergence to \ac{BNE} of NPGA Self-Play in auction games.
\end{abstract}
\acresetall{}

\maketitle


%----------------------------------------------------------------------
%%% TABLE OF CONTENTS
%----------------------------------------------------------------------
%\vspace{-5ex}
%\setcounter{tocdepth}{1}
%\tableofcontents
%\vspace{-5ex}



%*************************************************************
%*****    BODY TEXT
%*************************************************************

%% include files one by one

%----------------------------------------------------------------------
%%% INTRODUCTION
%----------------------------------------------------------------------



%----------------------------------------------------------------------
%%% PRELIMINARIES
%----------------------------------------------------------------------
\section{Preliminaries}
\label{sec:preliminaries}
% !TEX root = ./Main.tex
\subsection{Auctions as Bayesian Games}

    Let's start with some boilerplate text that shows how to use acronyms etc. Bold symbols in math can be written like this:
    $u \rightarrow \bu. \bA x = \boldb$ 
  
		A Bayesian game of incomplete information is described by a quintuple $G = (N,\mathcal A, V,F,u)$. 
    $N = \{1, \dots, n\}$ describes the set of players participating in the game. $\mathcal A = \mathcal A_1 \times \cdots \times \mathcal A_n$ is the set of possible action profiles, with $\mathcal A_i$ being the set of actions available to player $i \in N$. 
    In auctions, $\mathcal{A}_i$ are typically continuous sets of bids $\mathcal{A}_i \subset \R$. 
    $V = V_1 \times \cdots \times V_n$ is the set of \emph{type profiles}. 
		At the beginning of the game, each player $i$ is informed of her own type $v_i\in V_i$ only, thus the type constitutes private information. 
		Just as $\mathcal A_i$, the $V_i$ are (potentially continuous) subsets of $\mathbb{R}$. 
		Both the individual action and type spaces might be multidimensional in the case of multi-object auctions. 
    $F(v)$ defines a prior probability distribution over type profiles that is assumed to be common knowledge among all bidders in the Bayesian game. Each player's utility function is now determined by $u_i: \mathcal A \times V_i \rightarrow \R$, i.e. players' utilities depend on all other players' actions and only their own type.

    In this paper we consider \emph{sealed-bid auctions} on $J$ items with $N$ participating bidders. For sake of brevity, we limit the formal description to single-item auctions, but the extension to multiple items is straightforward, as described below. Thus for now, let $J=1$.  In each auction, a \emph{valuation profile} $v \sim F$ is drawn, and each bidder $i$ observes her \emph{private valuation} $v_i \in \R_{\geq 0}$ of the item. Now, $i$ submits a bid $b_i$ chosen according to some (possibly stochastic) strategy function $\pi_i: V_i \rightarrow \Delta\mathcal A_i$ that maps valuations to a probability distribution over possible actions. When $\pi_i$ is known to be deterministic, i.e. returns a single action or bid $b_i$ with probability 1, we will simply write $\pi_i(v_i)=b_i$. Throughout this paper, we denote by the index $-i$ a profile of types, actions or strategies for all players but player $i$. The auctioneer collects these bids, applies some auction mechanism that determines (a) an allocation $a \in \left\{0,1\right\}^{N}$, with $a_{i}=1$ iff player $i$ wins the item or 0 otherwiese, and (b) payments $p\in \R^N$ that the players have to pay to the auctioneer. Given some risk-constants $r_i > 0$, we model the bidder's utilities by the \emph{risk-adjusted payoffs}
    
    $$u_i = \left(a_i\cdot v_i - p_i\right)^{r_i}$$


    Here $r_i = 1$ corresponds to the risk-neutral case with quasilinear utility, whereas $r_i < 1$ indicates risk aversion. Most common auction mechanisms assign positive payments only when a player wins a desired item, i.e. $p_i > 0 \Leftrightarrow a_i=1$. In the case of multi-item combinatorial auctions, both private valuations and allocations are over \emph{bundles} of items and bids are mutlidimensional (depending on some \emph{bid language}). 

    In summary, each player is tasked with choosing a strategy $\pi_i(v_i)$, given knowledge about $v_i$ while not knowing other players' valuations beyond the fact that they have been drawn from one a prior distribution $F(v_{-i}\ \vert\ v_i)$, which is accessible to $i$ as $F(v)$ is common knowledge. In multi-object auctions such as combinatorial auctions, we might have multiple prior distributions. How to choose this strategy $\pi_i$ is a central question in Bayesian Game Theory. 

    
    In non-cooperative game-theory, \acp{NE} \citep{nashEquilibriumPointsNperson1950} are a central solution concept. Informally, in NE no agent has an incentive to deviate, given the equilibrium strategy of all other agents. Therefore, once a NE is found, it forms stable state. BNE extend the standard notion of the \ac{NE} in complete-information games by calculating the expected utility over the distribution of opponent valuations $v_{-i}$ in the ex-interim state in addition to just the opponent's strategies $\pi_{-i}$.  

    \Iac{BNE} is a strategy profile $\pi^*$ such that no player can improve her own ex-interim expected utility by deviating from it. Thus in \iac{BNE} the following holds for all players $i$ and all her possible types $v_i$ and strategies $\pi_i$:
    \begin{equation*}
        \exof{u_i\left(\pi_i(v_i), \pi^*_{-i}(v_{-i})\right) \given v_i} \leq \exof{u_i\left(\pi^*(v)\right) \given v_i}
        %\expect[v_{-i}\vert v_i]{u_i\left(\pi_i(v_i), \pi^*_{-i}(v_{-i})\right)} \leq \expect[v_{-i}\vert v_i]{u_i\left(\pi^*(v)\right)}
    \end{equation*}


    Some conditional expectations:

    \begin{equation*}
      \exof{a \given b}
    \end{equation*}

    \begin{equation*}
      \exofsub*{\mu}{\frac{a}{x} \given b}
    \end{equation*}

    Almost surely in text is \as and in math mode $\as$.
    Almost everywhere in text is \ae and in math mode $\ae$.

    In text mode $\exof{a \given b}$ end of bla $\exofsub{\mu}{\frac{a}{x} \given b}$
    
    Thus, in a BNE, every bidder's strategy maximizes his expected ex-interim utility given his beliefs about the state of nature and other players. Note that any strategy $\pi_i^*$ that is part of a \ac{BNE}-profile must also maximize $i$'s \emph{ex-ante} expected utility, which we will denote by $\bar u_i$:
    $$\bar{u}_i(\pi_i, \pi_{-i}) = \exof{u_i(\pi_i(v_i), \pi_{-i}(v_{-i}))}$$


    \begin{theorem}[Important Theorem]\label{thm:it}
      This is a theorem. It even has a numbered equation.
      \begin{equation}
        a^2 + b^2 = c^2
      \end{equation}
    \end{theorem}


\subsection{\acf{BNE} in concave games}

Nothing here yet, except this citation from \ac{RL}: \citet{silverDeterministicPolicyGradient2014}.



%*************************************************************
%*****    BIBLIOGRAPHY
%*************************************************************
\bibliographystyle{apalike}
%\bibliographystyle{spbasic}
\bibliography{IEEEabrv,Bibliography}



%*************************************************************
%*****    APPENDICES
%*************************************************************
\appendix



%----------------------------------------------------------------------
%%% APP: AUXILIARY
%----------------------------------------------------------------------
\section{Proofs omitted from the main text}
\label{appendix:proofs}
%----------------------------------------------------------------------
%%% APP: AUXILIARY
%----------------------------------------------------------------------
% !TEX root = ./Main.tex

%% Appendix, e.g. for proofs

\begin{Proof}[Proof of \cref{thm:it}]
This is some nice proof.
\end{Proof}


%---------------------------------------------------------
%%% APP: Acronyms
%----------------------------

% always include this to compile the acronyms, even if you
% don't want them to be listed.!
% if you do NOT want to print the acronyms, set the 
% [noprint] option on package import
% !TEX root = ./Main.tex
\makeAcronymSection
\begin{acronym}[ABCD]
    \acro{CCE}{coarse correlated equilibrium}
    \acrodefplural{CCE}[CCE]{coarse correlated equilibria}
    \acro{CE}{correlated equilibrium}
    \acrodefplural{CE}[CE]{correlated equilibria}
    % \acro{OMD}{online mirror descent}
    % \acro{OGD}{online gradient descent}
    % \acro{MDS}{martingale difference sequence}
    % \acro{LLN}{law of large numbers}
    % \acro{EW}{exponential weights}
    % \acro{APT}{asymptotic pseudotrajectory}
    % \acroplural{APT}{asymptotic pseudotrajectories}
    \acro{KKT}{Karush\textendash Kuhn\textendash Tucker}
    \acro{MSE}{mean squared error}
    \acro{RMSE}{root mean squared error}
    % \acro{DSC}{diagonal strict concavity}
    \acro{NE}{Nash equilibrium}
    \acrodefplural{NE}[NE]{Nash equilibria}
    \acro{BNE}{Bayesian Nash equilibrium}
    \acrodefplural{BNE}[BNE]{Bayesian Nash equilibria}
    \acro{RL}{reinforcement learning}
    \acro{NN}{neural network}
    \acro{MARL}{multi-agent reinforcement learning}
    % \acro{SD}[s/d]{source/destination}
    % \acro{ESS}{evolutionarily stable state}
    % \acro{MD}{mirror descent}
    % \acro{MA}{mirror ascent}
    % \acro{MMA}{multi-player mirror ascent}
    % %\acro{ML}{mirror-based learning}
    % \acro{DA}{dual averaging}
    % \acro{MDA}{multi-agent dual averaging}
    % \acro{SA}{stochastic approximation}
    \acro{VI}{va\-ri\-a\-tio\-nal inequality}
    \acrodefplural{VI}{va\-ri\-a\-tio\-nal inequalities}
    % \acro{VS}{variational stability}
    \acro{ES}{evolutionary strategy}
    % \acro{GNEP}{generalized Nash equilibrium problem}
    \acro{iid}[i.i.d.]{independent and identically distributed}%
\end{acronym}






\end{document}